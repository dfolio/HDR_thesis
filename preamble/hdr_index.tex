
% Acronym
\NewAcro{ENSI}{\French{École Nationale Supérieur d’Ingénieurs}}
\NewAcro{INSA}{\French{Institut National des Sciences Appliquée}}
%\NewAcro{CVLacr}
\def\CVL{\French{Centre Val de Loire}}

\NewAcro{PRISME}{\French{Laboratoire Pluridisciplinaire de Recherche en Ing\'enierie des Syst\`emes, M\'ecanique, \'Energ\'etique}\protect\GlsSeeMore{glos:PRISME}}
\newglossaryentry{glos:PRISME}{type=glg,
  name=PRISME Laboratory,
  description={The \PRISMEshort Laboratory is from \French{Université d'Orléa,s} and \INSAshort \CVL (EPRES 4229), \url{http://www.univ-orleans.fr/PRISME}.
    The \PRISMEshort laboratory seeks to carry out multidisciplinary research in the general domain of engineering sciences over a broad range of subject areas, including combustion in engines, energy engineering, aerodynamics, the mechanics of materials, image and signal processing, automatic control and robotics.  }
}
\NewAcro{NANOMA}{Nano-Actuators and Nano-Sensors for Medical Applications\protect\GlsSeeMore{glos:NANOMA}}
\newglossaryentry{glos:NANOMA}{type=glg,
  name=NANOMA project,
  description={The \NANOMAshort project is an European project funded under FP7-ICT-2007.3.6, Micro/nanosystems, coordinated by Professor Antoine Ferreira, \French{Université d'Orléans}. The NANOMA project aims at proposing novel controlled nanorobotic delivery systems which will be designed to improve the administration of drugs in the treatment and diagnosis of breast cancer.}
}

\NewAcro{ANR}{\French{Agence Nationale de la Recherche}}% National Agency for Research
\NewAcro{PIANHO}{Innovative Haptic Instrumental platform for 3D Nano-manipulation\protect\GlsSeeMore{glos:PIANHO}}
\newglossaryentry{glos:PIANHO}{type=glg,
  name=PIANHO project,
  description={The \PIANHOshort project is an ANR P3N (2009) project. The objective has been to design a micromanipulation platform capable of pick, hold and place nano-objects in the synchrotron radiation beam of the ESRF, Grenoble, France.}
}



\NewAcro{RMN}{r\'esonance magn\'etique nucl\'eaire}
\NewAcro{IRM}{imagerie  par r\'esonance magn\'etique}
\NewAcro{MRN}{\English{magnetic resonance navigation}}
\NewAcro{EMA}{actionnement \'electromagn\'etique}
\NewAcro{TMMC}{\English{Therapeutic magnetic microcarriers}}
\NewAcro{SPIO}{\English{superparamagnetic particles iron oxide}}

\NewAcro{MRA}{\English{Magnetic resonance angiography}}
\NewAcro{MRI}{\English{Magnetic Resonance Imaging}}
\NewAcro{IRIS}{Institut de Robotique et des Syst\`emes Intelligents, \url{http://www.iris.ethz.ch}}
\NewAcro{ETH}{Institut F\'ed\'eral de Technologie de Zurich (Suisse), \url{http://www.ethz.ch}}
\NewAcro{AMiR}{Division Microrobotics and Control Engineering de l'Universit\'e d'Oldenburg, Allemagne, dirig\'e par le professeur Sergej Fatikov, \url{http://www.amir.uni-oldenburg.de}}


\NewAcro{DDL}{degr\'es de libert\'e}
\NewAcro{RF}{radiofr\'equence}



\NewSymb[sort=c8]{Reynolds}{\ensuremath{\mathrm{R}_{e}}\xspace}{\NoUnit}{Nombre de Reynolds}
