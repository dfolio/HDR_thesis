% !TeX encoding = UTF-8
% !TeX spellcheck = en_US
% !TeX root = ../hdr_dfolio.tex

\Chapitre[chapI]{Context of Activities}

This chapter presents the various activities that I have conducted since my tenured position as  Associate Professor in 2008. It first start with an overall overview of my carrier.
This chapter oftenly refers to the appendices\;\RefAnnexeCV and \RefAnnexeRef that include my curriculum vit\ae{} and the list of my publications.

\section{Carrier overview}

\subsection{Doctorate degree and post-doctorate}
I defended my PhD in Robotics in 2007 within the  Robotics, Action, and Perception (RAP) group of the Laboratory for Analysis and Architecture of Systems (LAAS\footnote{LAAS: \url{http://www.laas.fr}}), CNRS\footnote{CNRS:  French National Center for Scientific Research, \url{http://www.cnrs.fr}}, under the supervision of Viviane Cadenat, Associate Professor at Paul Sabatier University in Toulouse. 
My PhD thesis dealt with the design of multi-sensor based control strategies allowing a mobile robot to perform vision-based tasks amidst possibly occluding obstacles.
Between 2007 and 2008, I joined the Lagadic team at Inria Rennes-Bretagne Atlantique as a postdoctoral fellow on sensory control for unmanned aerial vehicles. 
Especially, I have contributed to the design of a new on-line sensor self-calibration based on the sensor/robot interaction links \citep{2010_icra_kermorgant}%\CICL{2010_icra_kermorgant}.

\subsection{Tenured as Associate Professor}
In 2008, I was recruited as Associate Professor of the 61\up{st} CNU section at \ENSI Bourges, which is now  the \INSA Centre Val de Loire\footnote{INSA Centre Val de Loire (INSA CVL) was created in 2014 by the merge of \French{Ecole Nationale d'Ingénieurs du Val de Loire} (ENIVL) of Blois and ENSI of Bourges. In 2015, the \French{Ecole Nationale Supérieure de la Nature et du Paysage} (ENSNP) of Blois is integrated to INSA Centre Val de Loire. \url{http://www.insa-centrevaldeloire.fr}}. 
Since then, I have been regularly involved in the life of the institute. 
In particular, I contribute at a local level to the scientific animations (eg., organization of laboratory visits), transfer and training-research links. 
Thus, I regularly attend the international relations division by accompanying the different delegations of schools and  universities partners during their visits to \INSA Center Val Loire.
In March 2017, the direction of the \INSA Centre Val de Loire  given to me the mission of referent \enquote{racism and antisemitism}.

As senior lecturer, I am mainly involved in the development of electronics and electrical sciences teaching activities of the institute.
In particular, I have contributed to develop all of the teaching materials for the electronics and electrical sciences  courses and tutorials.
Since September 2014, I am in charge of the Nuclear Energy options of the 5\up{th} year (engineer's degree) of the Industrial Risk Control (MRI\footnote{MRI: \French{Ma\^itrise des Risques Industriels}}) department.

Furthermore, I achieve my research activities with the PRISME\footnote{PRISME Laboratory: \French{Laboratoire Pluridisciplinaire de Recherche en Ingénierie des Systèmes, Mécanique, Énergétique}. \url{https://www.univ-orleans.fr/prisme}} Laboratory in the Robotics axis of the IRAuS pole.
My research interests mainly concern modeling and control for nano and micro-robotics in a biomedical context.
In a first time